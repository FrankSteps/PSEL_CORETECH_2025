\documentclass[a4paper]{article} % A4, o padrão usado no Brasil.
\usepackage[brazilian]{babel} % Linguagem em português.
\usepackage{abnt} % Normas ABNT.
\usepackage[utf8]{inputenc} % UTF-8.
\usepackage[T1]{fontenc} % Permite fontes com mais glifos (letras).
% Fonte bonita que é usada até hoje.
% Veja: <https://tex.stackexchange.com/questions/147194/is-it-still-useful-to-load-the-lmodern-package>
\usepackage{lmodern}
\usepackage{amsmath, amsthm, amssymb} % Coisas de matemática.
\usepackage[table, xcdraw]{xcolor} % Definições fáceis de cor.
\usepackage[hyphens]{url} % Links.
\usepackage{bookmark} % Permite links fáceis entre múltiplos arquivos.
\usepackage{hyperref} % Referências.
\definecolor{bblue}{HTML}{0645AD} % Azul meio escuro.
\hypersetup{colorlinks, linkcolor=blue, urlcolor=bblue} % Links azuis.
% Subindo o título.
\usepackage{titling} % Informações de títulos melhores.
\setlength{\droptitle}{-6em}
\setlength{\parindent}{0pt}
\setlength{\parskip}{1em}
\usepackage[stretch=10]{microtype} % Ajuste de espaçamento.
\usepackage{hyphenat} % Configuração de hífens.
\usepackage{ragged2e} % Mais comandos de hífens.
\usepackage{listings} % Cores pra código.
\usepackage[explicit]{titlesec} % Estilos de título.
\usepackage{fvextra} % Melhorias para blocos de código.
\usepackage{tcolorbox} % Caixas legais que podem ser usadas para colocar código dentro.
\usepackage{calc} % Matemática em comandos.
\hypersetup{
  pdfauthor={Grupo 1},
  pdftitle={Problema C: Implementação de um interpretador feito em C++ para a arquitetura PicoQuickProcessor},
  pdfkeywords={},
  pdfsubject={},
  pdfcreator={TeX Live 2025},
  pdflang={English}}

\author{Grupo 1}
\date{\today}
\title{Problema C: Implementação de um interpretador feito em C++ para a arquitetura PicoQuickProcessor}

\makeatother
\begin{document}

\maketitle

\tableofcontents

\pagebreak

\section{Membros}

\begin{center}
  \begin{tabular}{ll}
    Nome & Curso\\
    \hline
    Franciso & Ciência da Computação\\
    Gabriel & Engenharia da Computação\\
    Guilherme & Ciência da Computação\\
    Rhuan Pablo & Engenharia da Computação\\
    Vinicíus & Engenharia da Computação\\
  \end{tabular}
\end{center}

\section{Introdução}

Texto aqui.

\section{Estruturas de Dados}

O projeto está organizado de uma maneira que se assimila ao paradigma de objetos, utilizando de \texttt{structs} para isso. Como legenda, saiba que, para todo campo de um \texttt{struct} que foi citado, em parênteses estará seu tipo.

\subsection{Memória}

Para a memória, um \texttt{struct} contendo dois campos foi utilizado.

\begin{enumerate}
\item \texttt{mem8} (\texttt{uint8\_t})
  \begin{description}
  \item Simula a memória de 256 bytes do processador.
  \end{description}
\item \texttt{size} (\texttt{uint32\_t})
  \begin{description}
  \item Indica a quantidade de memória a ser alocada pelo interpretador.
  \end{description}
\end{enumerate}

Os tipos usados vem do header \texttt{<stdint.h>}, e portanto possuem larguras fixas independentemente da implementação.

\subsection{Registradores}

Os registradores não estão organizados numa forma específica para eles, todavia, fazem parte do \texttt{struct} \texttt{Cpu}, que guarda as seguintes informações:

\begin{enumerate}
\item \texttt{pc} (\texttt{uint16\_t})
  \begin{description}
  \item Indica a quantidade de processos sendo executados pela CPU.
  \end{description}
\item \texttt{r[REG\_COUNT]} (\texttt{uint32\_t})
  \begin{description}
  \item Cria um \texttt{array} que servirá para simular os 16 registradores de propósito geral da arquitetura.
  \end{description}
\end{enumerate}

\subsection{Flags}

As flags, assim como os registradores, fazem parte da definição da CPU.

\section{Lógica Principal}

Texto aqui.

\section{Implementação de Instruções Complexas}

Texto aqui.

\section{Coleta de Dados}

Texto aqui.

\section{Desafios}

Texto aqui.

\section{Decisões}

Texto aqui.

\section{Referências}

(JAKE BOX (ED.). \textbf{Perfect Emacs Org Mode Exports to LaTeX – Straightforward Emacs}. 7 Mar. 2021. Disponível em: <\url{https://www.youtube.com/watch?v=0qHloGTT8XE}>. Acesso em: 13 nov. 2024)

\end{document}
